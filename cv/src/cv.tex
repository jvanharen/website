\documentclass[a4paper,10pt]{article}

\usepackage{marvosym}
\usepackage{fontspec}
\usepackage{amssymb}
\usepackage{pifont}
\usepackage{xunicode,xltxtra,url,parskip}
\usepackage[usenames,dvipsnames]{xcolor}
\usepackage[vmargin={1.5cm,1.5cm},hmargin={1.8cm,1.8cm}]{geometry}
\usepackage{supertabular}
\usepackage{titlesec}
\usepackage{array}
\usepackage{multirow}
\usepackage{hyperref}
\usepackage[absolute]{textpos}
%\usepackage{bibunits}
\usepackage{booktabs,tabularx}
\usepackage[style=src/custom,citestyle=numeric-comp,sorting=none,sortcites=false,
      autopunct=true,babel=hyphen,hyperref=true,abbreviate=false,
      backref=true,backend=biber,maxbibnames=99,
      date=year,isbn=false]{biblatex}

\DeclareSourcemap{\maps[datatype=bibtex]{\map{\step[fieldsource=author,match=Vanharen,final]\step[fieldset=keywords, fieldvalue=Vanharen]}}}

\def\UrlFont{\normalfont}
\DeclareFieldFormat{doi}{~\url{http://dx.doi.org/#1}}
\DeclareFieldFormat{url}{~\url{http://pages.saclay.inria.fr/julien.vanharen/#1}}
\DeclareFieldFormat*{pages}{#1}
\DeclareFieldFormat*{year}{(#1)}
\DeclareFieldFormat*{date}{(#1)}
\DeclareFieldFormat*{number}{(#1)}
\DeclareFieldFormat*{journaltitle}{\textit{#1}}
\DeclareFieldFormat*{booktitle}{#1}
\DeclareFieldFormat*{citetitle}{#1}
\DeclareFieldFormat*{title}{#1}

\renewbibmacro{in:}{}
\renewbibmacro{mkbibquote}{}

\defbibheading{bibempty}{}
\addbibresource{src/biblio.bib}
\include{abbr_journal}

\definecolor{linkcolour}{RGB}{0,129,168}
\hypersetup{colorlinks,breaklinks,urlcolor=linkcolour,linkcolor=linkcolour}
\defaultfontfeatures{Mapping=tex-text}
\setmainfont[Path = fonts/, SmallCapsFont = Fontin-SmallCaps.otf, BoldFont = Fontin-Bold.otf, ItalicFont = Fontin-Italic.otf]{Fontin.otf}
\titleformat{\section}{\Large\scshape\raggedright}{}{0em}{}[\titlerule]
\titlespacing{\section}{0pt}{3pt}{3pt}
\setlength{\TPHorizModule}{30mm}
\setlength{\TPVertModule}{\TPHorizModule}
\textblockorigin{2mm}{0.65\paperheight}
\setlength{\parindent}{0pt}

\begin{document}
\pagestyle{empty}
\par{\begin{center}
        {\Huge Julien \textsc{Vanharen}}
        \smallskip\par{\Large Ph.D. in Computational Fluid Dynamics} \end{center}}
\smallskip\par

\section{Personal Data}
\begin{tabular}{rl}
    \textsc{Place and Date of Birth:} & Soissons, France | 31 July 1989                                        \\
    \textsc{Citizenship:}             & French                                                                 \\
    \textsc{Phone:}                   & +33 (0) 6 60 74 88 16                                                  \\
    \textsc{Email:}                   & \href{mailto:julien.vanharen@gmail.com}{julien.vanharen@gmail.com}     \\
    \textsc{Website:}                 & \url{http://pages.saclay.inria.fr/julien.vanharen/}                    \\
    \textsc{Address:}                 & 22-24, avenue Marcellin Berthelot, Apt. A302, 92320 Ch\^atillon France \\
\end{tabular}

\section{Work Experience}
\begin{tabular}{r|p{15cm}}

    \textsc{May 2017} & Post-Doc at \textsc{INRIA} in Saclay, France                                                                                               \\
                      & \emph{Time-Accurate Anisotropic Mesh Adaptation for Fluid Structure Interaction Simulations}                                               \\
                      & \footnotesize{Implementation and validation of the Finite Element Method for the unsteady linear elasticity equations.
        Coupling with a code solving the Euler equations based on the Finite Volume Method. Development of error estimates
        for fluid structure interaction applied to unsteady anisotropic mesh adaptation.
        Project RAPID funded by DGA (Direction G\'en\'erale de l'Armement). }                                                                                      \\
    \multicolumn{2}{c}{}                                                                                                                                           \\

    \textsc{Aug 2016} & Argonne Training Program on Extreme-Scale Computing (ATPESC 2016) in St. Charles, IL, USA                                                  \\
                      & \footnotesize{Intensive, two-week training on the key skills, approaches, and tools
        to design, implement, and execute computational science and engineering applications
        on current high-end computing systems and the leadership-class computing systems of the future.}                                                           \\
    \multicolumn{2}{c}{}                                                                                                                                           \\

    \textsc{May 2014} & \href{http://pages.saclay.inria.fr/julien.vanharen/files/phd.pdf}{Ph.D. Thesis} at \textsc{AIRBUS} \& \textsc{CERFACS} in Toulouse, France \\
    \textsc{Apr 2017} & \emph{High-Order Numerical Methods For Unsteady Flows Around Complex Geometries}                                                           \\
                      & \footnotesize{Several numerical methods \& codes are investigated for industrial applications, among them the coupling with a
        nonconforming grid interface of high-order schemes for structured and unstructured
        zones in \href{http://elsa.onera.fr}{\emph{elsA}},
        the Spectral Difference Method in \href{http://www.cerfacs.fr/~puigt/jaguar.html}{\emph{\textsc{Jaguar}}}
        and the Lattice Boltzmann Methods in \href{http://www.prolb-cfd.com}{\emph{\textsc{ProLB}}}.}                                                              \\
    \multicolumn{2}{c}{}                                                                                                                                           \\

    \textsc{Jan 2014} & Study Engineer at \textsc{CERFACS} in Toulouse, France                                                                                     \\
    \textsc{Mar 2014} & \emph{Computational Fluid Dynamics \& High-Order Numerical Methods}                                                                        \\
                      & \footnotesize{URANS computations on unstructured meshes \& stability of the Spectral Difference Method.}                                   \\
    \multicolumn{2}{c}{}                                                                                                                                           \\

    \textsc{Apr 2013} & Research Intern at \textsc{CERFACS} in Toulouse, France                                                                                    \\
    \textsc{Sep 2013} & \emph{Numerical Methods}                                                                                                                   \\
                      & \footnotesize{Theorerical and numerical analysis of nonconforming grid interface between structured blocks
        for Large Eddy Simulation. Implementation inside the \href{http://elsa.onera.fr}{\emph{elsA}}
        software of a Riemann solver at the interface.}                                                                                                            \\
    \multicolumn{2}{c}{}                                                                                                                                           \\

    \textsc{Jun 2012} & Engineering Intern at \textsc{AIRBUS} in Bremen, Germany                                                                                   \\
    \textsc{Aug 2012} & \emph{Handling Qualities}                                                                                                                  \\
                      & \footnotesize{A340-600 MSN360 flight test analysis: effect of the vortex generators on rudder efficiency.}                                 \\
    \multicolumn{2}{c}{}                                                                                                                                           \\

    \textsc{Jul 2011} & Engineering Intern at \textsc{AIRBUS} in Toulouse, France                                                                                  \\
    \textsc{May 2012} & \emph{Methods \& Tools}                                                                                                                    \\
                      & \footnotesize{Unsteady aerodynamics: gust and aileron deflection effect on aircraft behaviour.
        Extension and improvement of CFD methods to predict flutter thanks to dynamic coupling.
        Selected to be part of the \emph{EADS Junior Programme}.}
\end{tabular}

\section{Education}
\begin{tabular}{r|p{15cm}}
    \textsc{Aug} 2009 & \textbf{ISAE-Supa\'ero} in Toulouse, France.                                                                               \\
    \textsc{Nov} 2013 & ``Dipl\^{o}me d'ing\'{e}nieur'' in \textsc{Aerospace Engineering}.                                                         \\
                      & Major: Aerodynamics.                                                                                                       \\
                      & \textsc{Research Master} in fluid dynamics.                                                                                \\
    \multicolumn{2}{c}{}                                                                                                                           \\

    \textsc{Sep} 2007 & \textbf{Lycée Faidherbe} in Lille, France.                                                                                 \\
    \textsc{Jun} 2009 & Undergraduate studies in a national preparatory program for entrance into France state run graduate school of engineering. \\
    \multicolumn{2}{c}{}                                                                                                                           \\
\end{tabular}

\section{Teaching}
\begin{tabular}{r|p{15cm}}
    \textsc{May} 2016 & Training session at \textbf{CERFACS} in Toulouse, France.                                \\
                      & Fundamentals to understand and analyze high fidelity compressible Large Eddy Simulation. \\
    \multicolumn{2}{c}{}                                                                                         \\

    \textsc{Jun} 2019 & \textbf{ENSTA} in Palaiseau, France.                                                     \\
                      & Incompressible fluid mechanics (MF102).                                                  \\
\end{tabular}

\section{Skills}
\begin{tabular}{rl}
    \textsc{Systems}        & UNIX, Linux, Mac OS X, HPC.                                                                            \\
    \textsc{Simulation}     & \href{http://pages.saclay.inria.fr/frederic.alauzet/}{Wolf}, \href{http://elsa.onera.fr}{\emph{elsA}},
    \href{http://gpuigt.free.fr/jaguar.html}{\emph{\textsc{Jaguar}}},
    \href{http://www.prolb-cfd.com}{\emph{\textsc{ProLB}}}.                                                                          \\
    \textsc{Programming}    & C/C++, F90, Python, OpenMP, MPI, CMake, \href{http://cgns.github.io}{CGNS}.                            \\
    \textsc{Meshing}        & \href{http://www.meshgems.com}{GHS3D}, \href{https://pyamg.saclay.inria.fr/}{Feflo.a},
    \href{https://www.centaursoft.com}{Centaur}, \href{https://www.beta-cae.com/ansa.htm}{Ansa}, \href{http://gmsh.info}{Gmsh}.      \\
    \textsc{Automation}     & {\fontfamily{cmr}\selectfont \LaTeX}, Keynote, Microsoft Office Suite.                                 \\
    \textsc{Post-Treatment} & \href{http://vizir.inria.fr/}{Vizir}, \href{http://cerfacs.fr/antares}{Antares},
    \href{https://www.paraview.org}{Paraview}, \href{https://www.tecplot.com}{Tecplot}.                                              \\
    \textsc{Reviewer}       & Journal of Computational Physics.
\end{tabular}

\section{Languages}
\begin{tabular}{rl}
    \textsc{French}  & Native proficiency.                                    \\
    \textsc{English} & Full working proficiency: \textsc{TOEFL ITP} 550 (B2). \\
    \textsc{German}  & Good working knowledge.                                \\
    \textsc{Polish}  & Basic communication skills.
\end{tabular}

\nocite{*}
\section{Publications}
\newrefcontext[sorting=ynt]
\printbibliography[heading=none, sorting=ynt, type=article]
\section{Conferences}
\newrefcontext[sorting=ynt]
\printbibliography[heading=none, sorting=ynt, type=inproceedings]

\section{References}
\begin{tabular}{l}
    Dr. HdR. Guillaume Puigt                                                     \\
    Ma\^itre de recherche 1 \textsc{ONERA}                                       \\
    \textsc{ONERA} - The French Aerospace Lab                                    \\
    Unit\'e HEAT - D\'epartement DMPE                                            \\
    2, avenue Edouard Belin                                                      \\
    BP 74025 - 31055 Toulouse Cedex 4, France                                    \\
    \ding{37} +33 (0) 5 62 25 29 40                                              \\
    \ding{41} \href{mailto:guillaume.puigt@onera.fr}{guillaume.puigt@onera.fr}   \\
    \\
    Dr. HdR. Fr\'ed\'eric Alauzet                                                \\
    Directeur de recherche \textsc{INRIA}                                        \\
    \textsc{INRIA} Saclay - \^Ile-de-France                                      \\

    Bâtiment Alan Turing                                                         \\
    1 rue Honoré d'Estienne d'Orves                                              \\
    91120 Palaiseau, France                                                      \\
    \ding{37} +33 (0) 1 77 57 80 91                                              \\
    \ding{41} \href{mailto:frederic.alauzet@inria.fr}{frederic.alauzet@inria.fr} \\
    \\
    Dr. Adrien Loseille                                                          \\
    \textsc{INRIA} Saclay - \^Ile-de-France                                      \\
    Bâtiment Alan Turing                                                         \\
    1 rue Honoré d'Estienne d'Orves                                              \\
    91120 Palaiseau, France                                                      \\
    \ding{37} +33 (0) 1 77 57 80 15                                              \\
    \ding{41} \href{mailto:adrien.loseille@inria.fr}{adrien.loseille@inria.fr}
\end{tabular}

\end{document}